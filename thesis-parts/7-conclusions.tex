% ! TeX root = ../bachelor-thesis.tex

\chapter*{Conclusioni}
\addcontentsline{toc}{chapter}{Conclusioni}

In questo elaborato si è cercato d'illustrare al meglio le caratteristiche di
un sistema che potesse soddisfare i requisiti funzionali preposti. Il risultato
è stato nel complesso adeguato e soddisfacente, ma possono essere fatte alcune
considerazioni finali.

In primo luogo, il sistema è ancora in fase prototipale e saranno necessari
alcuni raffinamenti. Nello specifico, per quanto riguarda il sistema domotico,
sarà necessario cercare di migliorarne la reattività. Ciò dipende in gran parte
dalle tecnologie utilizzate, quindi si dovrà mantenere il sistema aggiornato,
in modo da adottare le nuove funzionalità e i miglioramenti che queste
offriranno. Per quanto riguarda le \textit{skill} implementate, sarà invece
necessario continuare a perfezionarne il modello di dialogo, rendendo sempre
più agevoli e interessanti le interazioni con Alexa.

In secondo luogo, il sistema dovrà essere verificato non solo in ambiente
chiuso, ma anche attraverso il coinvolgimento di un campione del target
indicato. Questo risulterà necessario per dimostrare l'effettiva adeguatezza
del sistema, soprattutto per gli esercizi cognitivi implementati.

In futuro, il progetto potrà essere esteso ed essere integrato in un contesto
socio-sanitario, nel quale potrà sperabilmente assistere persone con disabilità
o difficoltà cognitive a raggiungere una maggiore indipendenza.

In conclusione, si spera che questo progetto possa essere ultimato, producendo
tutti i risultati desiderati, e che possa concretamente assistere altre
persone, offrendo loro l'opportunità di ottenere una maggiore libertà.
