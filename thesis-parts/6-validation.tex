% ! TeX root = ../bachelor-thesis.tex

\chapter{Validazione del sistema}
\label{ch:Chapter6}

In questo capitolo, si descriveranno i metodi utilizzati per validare
l'applicazione, ovvero i test che sono stati eseguiti per verificarne la
correttezza e l’adeguatezza rispetto i requisiti preposti.

\section{Metodi di validazione}
\label{sec:Section6.1}

Al tempo della scrittura di questo elaborato, viste anche le limitazioni legate
all'emergenza \textit{COVID-19}, non era ancora stato possibile eseguire delle
prove che coinvolgessero il target dell'applicazione. Questi test sugli utenti
finali sono necessari ancor di più nel contesto della \textit{gamification},
quindi saranno sicuramente eseguiti in futuro.

Per il momento, nondimeno, sono stati eseguiti numerosi \textit{friendly-trial}
in ambiente chiuso all’interno del CRA di Bologna. Nello specifico, sono state
eseguite delle prove che verificassero:
\begin{itemize}
  \item[--] La \textit{\textbf{possibilità di controllare i dispositivi
            domotici attraverso OpenHAB}}; per questo è stata utilizzata
        l’interfaccia grafica costruita, accertandosi che a un comando
        dell'utente corrispondesse una reazione adeguata del sistema
        domotico (ad esempio, in riferimento alle figure
        \ref{fig:figure5.2} e \ref{fig:figure5.3}, alla pressione del
        pulsante \textit{light\_switch}, la lampadina \textit{lightbulb1}
        si deve accendere o spegnere);
  \item[--] La \textit{\textbf{possibilità di controllare i dispositivi
            domotici con dei comandi vocali attraverso Alexa}}; per questo è
        stata utilizzata la skill di OpenHAB attraverso Alexa, accertandosi che
        a un comando vocale dell'utente corrispondesse una reazione adeguata
        del sistema domotico (ad esempio, in riferimento alle figure
        \ref{fig:figure5.2} e \ref{fig:figure5.3}, dicendo ad Alexa
        \textit{"Accendi / Spegni la Lampadina Hue"}, la lampadina
        \textit{lightbulb1} si deve accendere o spegnere);
  \item[--] La \textit{\textbf{capacità di OpenHAB di notificare l’utente di
            certi eventi relativi ai dispositivi domotici connessi}}; per
        questo sono stati generati degli eventi nel sistema domotico,
        accertandosi che questi fossero comunicati correttamente attraverso
        Alexa (ad esempio, in riferimento alla figura \ref{fig:figure5.4},
        Alexa deve comunicare quando la lampadina \textit{lightbulb1} viene
        accesa o spenta);
  \item[--] L'\textit{\textbf{adeguatezza delle skill nell’esercitare le
            capacità cognitive dell’utente}}; per questo si è lasciato giocare
        alcuni utenti che non fossero direttamente coinvolti
        nell'implementazione delle \textit{skill}, prendendo atto delle
        recensioni e del feedback ricevuto, così da effettuare cambiamenti che
        ne migliorassero l'usabilità.
\end{itemize}
Il funzionamento del sistema è stato testato da alcuni membri del personale del
CRA di Bologna. Nello specifico, i risultati ottenuti sul sistema domotico e
sulle \textit{skill} di Alexa sono stati approvati da alcuni componenti del
team multidisciplinare del CRA, tra cui la neuropsicologa \textit{Dott.ssa
  Arianna Gherardini}, e i correlatori \textit{Ing. Massimiliano Malavasi},
coordinatore del CRA, e \textit{Dott.ssa Laura Bugo}, sviluppatrice software.

\section{Risultati ottenuti}
\label{sec:Section6.2}

Di seguito, saranno riportati e commentati i risultati ottenuti dai test
eseguiti sul sistema domotico e sulle \textit{skill} implementate.

\subsection{Il sistema domotico}
\label{subsec:Section6.2.1}

Il funzionamento del sistema domotico si è rilevato adeguato e soddisfacente.
Tuttavia, è possibile fare alcuni commenti sulle responsività constatate nelle
diverse interazioni rese disponibili all'utente.

L'\textit{interazione attraverso l'interfaccia grafica} è quella con minor
latenza. Ciò ha senso se si tiene conto del fatto che i comandi inviati tramite
interfaccia grafica sono inviati a OpenHAB senza intermediari. Maggiore è
invece la latenza delle \textit{interazioni vocali attraverso Alexa}. In questo
caso, i comandi devono essere interpretati dal cloud di Alexa, per questo
trascorrono un paio di secondi prima che l'utente possa osservare un riscontro
sul sistema domotico. Considerando che lo scopo dell'integrazione con Alexa era
in primo luogo di fornire ulteriori strumenti accessibili per controllare il
proprio sistema domotico, i tempi di risposta ottenuti sono stati valutati come
accettabili. Altrettanta latenza si è stata constatata nella \textit{ricezione
delle notifiche} sugli eventi accaduti nel sistema domotico. Questa funzione
era stata pensata per ricordare all'utente alcuni compiti da svolgere, a orari
prestabiliti, in base allo stato del sistema domotico (come ricordare
all'utente di chiudere le finestre d'inverno prima di dormire, se ce n'è
qualcuna aperta). Qualche secondo di differenza rispetto a tali orari, non
influenza il risultato finale, quindi anche in questo caso si è ritenuto il
sistema soddisfacente.

\subsection{Number List Game}
\label{subsec:Section6.2.2}

Riguardo alla \textit{skill} Number List Game, sono stati ricevuti alcuni
feedback costruttivi, soprattutto riguardo ad alcuni problemi di progettazione.
Nel complesso, è stato un buon primo tentativo d'implementazione di una
\textit{skill} per Alexa.

I feedback ricevuti riguardano soprattutto il flusso di dialogo. Infatti, la
\textit{skill} richiede una conferma ogni volta che l'utente invia la risposta
al quesito precedentemente comunicato da Alexa. Considerando che, nelle
interazioni con una \textit{skill}, ovvero con il cloud, esiste anche una certa
latenza, ciò può diventare abbastanza tedioso per il giocatore. Nelle
\textit{skill} progettate successivamente, si è dunque deciso di snellire il
modello di dialogo, raggiungendo gli stessi risultati attraverso meno
interazioni.

\subsection{Category Game}
\label{subsec:Section6.2.3}

La \textit{skill} Category Game ha ricevuto feedback molto positivi. Non
significa però, che non possano ancora essere migliorati alcuni suoi aspetti.

Il problema principale è la mancanza di contenuti. In effetti, la
\textit{skill} necessita di una maggiore varietà di categorie e di parole
conosciute: capita spesso che sia richiesto all'utente di categorizzare la
stessa parola in un breve periodo di tempo, oppure di fare partite consecutive
in cui siano coinvolte le stesse categorie. Siccome le categorie e le parole
sono definite in modo statico, una soluzione potrebbe essere l'aggiornamento
dei file che le contengono, o ancora meglio, se possibile, richiederle a
servizi terzi che già gestiscono dati simili.

\subsection{Labyrinth Game}
\label{subsec:Section6.2.4}

Labyrinth Game è stato un esperimento che ha avuto abbastanza successo. In fase
di progettazione, si pensava infatti che sarebbe stato troppo difficile per
l'utente seguire il flusso di dialogo del gioco e pertanto interagire con la
\textit{skill}. Tuttavia, si è scoperto invece che ci si abitua velocemente
alla struttura con cui si riferiscono le informazioni nel gioco. Dunque, già
all'interno della prima partita, le difficoltà iniziali vengono riscontrate
sempre meno.

Questi risultati sono stati ottenuti dopo alcune fasi di raffinamento, in cui
lo scopo principale era di bilanciare il dialogo tra l'utente e Alexa: in
questo tipo di gioco, l'utente riceve molte più informazioni rispetto a quelle
che fornisce. Per agevolare la concentrazione dell'utente, si è quindi cercato
di essere il più possibile concisi e schematici nel riferirgli le informazioni
necessarie per giocare, così da rendere il dialogo più interattivo.
