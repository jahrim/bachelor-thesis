% ! TeX root = ../bachelor-thesis.tex

\chapter{Introduzione}

L’uomo ha da sempre cercato di sopperire ai propri limiti, inventando e
costruendo strumenti che lo aiutassero nel soddisfare i propri bisogni di
libertà. Recentemente, uno di questi strumenti è stato la domotica, che, tra le
altre cose, si propone di facilitare la vita in ambito domestico, permettendo
di velocizzare o addirittura automatizzare, quelle piccole azioni quotidiane
che possono rubare via del tempo, oggi più che mai prezioso. Non è tutto però.
Infatti è proprio attraverso la domotica che molte persone ora vedono davanti a
sé una prospettiva d'indipendenza, nella quale quelle azioni quotidiane, prima
per loro impraticabili, diventano finalmente accessibili.

Con il passare del tempo, l’accessibilità ha acquisito sempre maggiore
importanza. Più precisamente, per accessibilità s’intende la capacità di un
servizio di essere fruibile dal maggior numero di persone possibile.
L’accessibilità è quindi cruciale per includere le persone, ma anche per
assicurare il nostro futuro. Infatti, man mano che si invecchia, sempre più
strumenti diventano sempre meno accessibili.

Ormai la domotica è un concetto ben consolidato e sta cominciando a diffondersi
su larga scala, permeando nelle case delle persone, a cui sta diventando sempre
più familiare. Per questo motivo, si presta come ottimo punto di appoggio per
lo sviluppo di servizi integrativi, che possano estendere le funzionalità del
proprio sistema domotico.

In questo contesto, in collaborazione con la AUSL \cite{AUSL} e il CRA
\cite{CRA} di Bologna, sotto la guida del coordinatore del CRA \textit{Ing.
Massimiliano Malavasi}, la neuropsicologa \textit{Dott.ssa Arianna Gherardini}
e la sviluppatrice software \textit{Dott.ssa Laura Bugo}, si è pensato di
progettare un servizio integrativo a un impianto domotico, che consenta di
mantenere in allenamento le proprie capacità cognitive, oltre a fornire
assistenza nello svolgimento delle attività di vita quotidiana. Questa suite è
stata pensata soprattutto come supporto per gli anziani o per le persone con
difficoltà cognitive, ma potrà anche svolgere una funzione di agevolazione e
intrattenimento per i giovani. Più in particolare, si vorrebbe applicare la
suite a un contesto socio-sanitario per assistere persone con disabilità e/o
con difficoltà cognitive e monitorarne i progressi durante il loro percorso
psico-educativo, sotto la supervisione di personale qualificato. La suite
prevederà anche un’assistente vocale che avrà sia la funzione di aumentare
l’accessibilità dell’impianto domotico, sia di permettere l’accesso al servizio
integrativo.

Questa tesi illustrerà come è stato sviluppato questo progetto dalla fase di
analisi del problema fino ai primi risultati prodotti. Sarà quindi descritta
una panoramica generale dei concetti necessari a comprendere le tecnologie
coinvolte e l’obiettivo preposto. Successivamente, sarà analizzato il problema,
delineando quelli che saranno i requisiti del sistema. Sarà poi proposta una
possibile soluzione concettuale, non vincolata a dei servizi specifici, che
descriverà la natura del problema. A questa seguirà una soluzione concreta,
basata su servizi oggi esistenti, di cui se ne motiverà l’adeguatezza,
illustrandone vantaggi e svantaggi. Infine, sarà presentata l'implementazione
di un prototipo e commentati i risultati dei test eseguiti durante la fase di
validazione.
